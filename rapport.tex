\documentclass[a4paper,titlepage]{report}
\usepackage[T1]{fontenc}
\usepackage[utf8]{inputenc}
\usepackage[francais]{babel}
\usepackage{lmodern} % police véctorielle
\usepackage{pdfpages}
\usepackage{graphicx}
\usepackage{subfig}
\usepackage[hidelinks]{hyperref}
%\usepackage{url}
%\usepackage{array}
\usepackage{listings}
\usepackage{color}
\usepackage{amsmath}
% definition des marges
\usepackage[top=2.5cm,bottom=2.5cm,right=2.5cm,left=3.5cm]{geometry}
\usepackage{listingsutf8}
\usepackage{lscape}
\pagestyle{headings}

%retrait du numero des chapitre dans les sections
\makeatletter
\renewcommand{\thesection}{\@arabic\c@section}
\makeatother

\title{Rapport du projet de conception formelle
	\begin{center}
		\textsc{Rodin 2014}
	\end{center}
}
\author{Jordan Aupetit, Salmane Bah, Timothée Sollaud, Bruno Thiao-Layel\\
	\\
	\\ \textsc{Université de Bordeaux}
}
\begin{document}
	\maketitle
	%\newpage
	\tableofcontents
	\newpage
	\thispagestyle{empty}	
	\newpage
	\pagenumbering{arabic} % Arabic page numbers from now on
% === Contenu =======================================================
\markboth{}{}
\section{Introduction}
	Dans le cadre de l'unité d'enseignement Conception Formelle du Master Informatique Génie Logiciel de l'université de Bordeaux, il nous a été proposé de formaliser le fonctionnement d'un bus arrivant à un arrêt.



\section{ContextUsagers}
	Nous avons défini un contexte \textbf{Usagers}, afin de représenter l'ensemble fini des usagers et la constante \textbf{BusCapacity}, représentant la capacité du bus.\\

\section{MachineBus0}
	La \textbf{MachineBus0} représente notre machine de base. Celle-ci assure la cohérence de l'enchaînement des évènements et le respect de la règle de civisme proposée par le sujet. Ainsi, les usagers montent dans le bus qu'après que tous les passagers désirant descendre l'aient fait.\\
	
	\subsection{Les ensembles}
		\begin{description}
			\item[usagers\_a\_l\_arret] est un sous-ensemble des usagers représentant les usagers attendant au prochain arrêt.\\
			\item[passagers] est un sous-ensemble des usagers représentant les passagers du bus. Le cardinal de passager ne peut donc pas dépasser la capacité du bus.\\
			\item[passagers\_descendre] est un sous-ensemble des passagers représentant les passagers souhaitant descendre du bus au prochain arrêt.\\
		\end{description}
		
	\subsection{Les évènements}
		Pour définir les gardes de nos évènements, nous avons utilisé le booléen \textbf{bus\_roule}, qui est à vrai si le bus est en mouvement et à faux si le bus est à un arrêt.\\
		
		\begin{description}
			\item[bus\_arrive] est l'évènement qui représente le fait que le bus se stoppe à un arrêt.
			
			Cet évènement peut se produire uniquement si le bus roule. 
			
			L'effet de l'évènement est simplement d'arrêter le bus.\\
		
			\item[bus\_repart] est l'évènement qui représente le fait que le bus repart d'un arrêt.
			
			Cet évènement peut se produire uniquement si le bus est arrêté. De plus, le bus ne doit repartir qu'une fois que tous les passagers souhaitant descendre l'ont fait et que les usagers attendant à l'arrêt soient tous montés ou que le bus soit plein. \\
			
			Dans notre première implémentation, nous avions fait en sorte que le bus puisse s'arrêter puis repartir puis s'arrêter à nouveau dans une boucle infinie. Cela permettait au bus de pouvoir parcourir une multitude d'arrêts, en faisant monter et descendre un certain nombre de passager à chaque fois.\\
			
			Dans notre seconde implémentation, afin de ne pas avoir une exécution infinie et ajouter un variant, nous avons décidé que le bus ne pourrait que s'arrêter et repartir une seule fois. De ce fait, l'évènement \textbf{bus\_repart} qui précisait que le bus roulait à nouveau, indique maintenant que la simulation se termine. Donc, lorsqu'elle est terminée, il n'est plus possible de faire aucun évènement. Nous obtenons donc une deadlock afin d'obtenir une terminaison.
			


			\item[passager\_desc] est l'évènement qui représente le fait qu'un passager descende du bus.
			
			Cet évènement peut se produire uniquement si le bus est arrêté et qu'il y a au moins un passager qui souhaite descendre.
			
			L'effet de l'évènement est de repasser le passager descendu en simple usager. \\
			
			\item[usager\_monte] est l'évènement qui représente le fait qu'un passager monte dans le bus.
			
			Cet évènement ne peut se produire que si le bus est arrêté, qu'au moins un usager souhaite monter et que tous les passagers voulant descendre l'ont fait.
			
			L'effet de cet évènement est de déplacer l'usager de l'ensemble des usagers attendant à l'arrêt vers l'ensemble représentant les passagers du bus.\\
		
			\item[usager\_arrive] est l'évènement représentant un usager arrivant à un arrêt, dans l'espoir de prendre le prochain bus.
			
			Cet évènement peut se produire à tout moment, tant qu'il reste au moins un usager qui n'est ni à un arrêt, ni déjà passager du bus.
			
			L'effet de l'évènement est simplement d'ajouter l'usager à la liste des usagers souhaitant monter.\\
			
			\item[passager\_veut\_desc] est l'évènement qui représente le fait qu'un passager du bus veuille descendre au prochain arrêt.
			
			Cet évènement peut se produire uniquement si le bus roule et qu'il reste des passagers n'ayant pas déjà décidé de descendre.
			
			L'effet de l'évènement est d'ajouter le passager à l'ensemble des passagers souhaitant descendre.\\
		\end{description}

	\subsection{Le variant}
		Maintenant que notre machine se termine nous avons réussi à mettre en place un variant. Pour ce faire, nous avons rajouté un contexte qui permet de convertir un booléen en entier qui va nous servir dans notre variable. Ce contexte s'appelle \textbf{BoolToInt}. Nous avons choisis comme variant le nombre de passagers voulant descendre, auquel on ajoute un entier représentant l'état de la simulation (0 pour terminée, sinon 1). Nous avons rajouté cet entier afin que le variant ait la valeur nulle dès que la simulation est terminée. Dans le cas contraire, le variant aurait eu la valeur nulle avant de repartir de l'arrêt. Nous avons réussi à prouver ce variant, ce qui démontre que notre machine se termine quoi qu'il arrive. La mise en place de celui-ci nous a posé des difficultés et nous a obligé de passer de notre première implémentation où notre bus roulait d'arrêt en arrêt dans une boucle infinie, à une implémentation où la simulation se termine dès que le bus repart du premier arrêt. Nous avons simplifié notre modèle pour que la preuve de terminaison soit réalisable.
		
	\subsection{Cas de l'arrêt du bus}
		Dans notre première implémentation, nous avions rajouté une garde sur l'évènement \textbf{bus\_arrive} permettant de préciser au bus de ne s'arrêter que lorsque c'était nécessaire. En effet, il ne pouvait s'arrêter que si des passagers voulaient descendre ou si des usagers voulaient monter et que le bus n'était pas remplit. On peut trouver dans l'évènement cette garde en commentaire. Nous l'avons finalement enlevée car celle-ci compliquait nos preuves et que ce n'est pas demandé dans le sujet. Dorénavant notre bus peut s'arrêter à tout moment.
	
\newpage
\section{MachineBus1}
	La \textbf{MachineBus1} est le premier raffinement de la machine de base \textbf{MachineBus0}. Elle intègre la gestion d'usagers prioritaires.\\
		
	\subsection{Les ensembles}
		Dans cette machine, les ensembles décrits précédemment n'ont pas été modifiés, excepté \textbf{usagers\_a\_l\_arret}, qui se décompose à présent en deux ensembles afin de gérer des usagers prioritaires :
		 	
		\begin{description}
			\item[usagers\_a\_l\_arret\_prioritaire], qui représente la partition des usagers prioritaires attendant au prochain arrêt.
			\item[usagers\_a\_l\_arret\_non\_prioritaire], qui représente la partition des usagers non prioritaires attendant au prochain arrêt.\\
		\end{description}
				
	\subsection{Les évènements}
		On retrouve dans ce raffinement la plupart des évènements de la machine d'origine, dans lesquels l'ensemble \textbf{usagers\_a\_l\_arret} a été remplacé par l'union des ensembles \textbf{usagers\_a\_l\_arret\_prioritaire} et \textbf{usagers\_a\_l\_arret\_non\_prioritaire}.\\
		
		L'évènement \textbf{usager\_arrive} a été décomposé en deux évènements, afin de gérer les priorités :
		\begin{description}
			\item[usager\_arrive\_prioritaire] est l'évènement représentant un usager prioritaire arrivant à un arrêt, dans l'espoir de prendre le prochain bus.
			
			Cet évènement peut se produire à tout moment, tant qu'il reste au moins un usager qui n'est ni en attente à un arrêt, ni déjà passager du bus.
			
			L'effet de l'évènement est simplement d'ajouter l'usager à la liste des usagers prioritaires souhaitant monter.
			
			\item[usager\_arrive\_non\_prioritaire] est l'évènement représentant un usager non prioritaire arrivant à un arrêt, dans l'espoir de prendre le prochain bus.
			
			Cet évènement peut se produire à tout moment, tant qu'il reste au moins un usager qui n'est ni en attente à un arrêt, ni déjà passager du bus.
			
			L'effet de l'évènement est simplement d'ajouter l'usager à la liste des usagers non prioritaires souhaitant monter.\\
		\end{description}
		

		De même, afin de gérer les priorités, l'évènement \textbf{usager\_monte} a été décomposé en deux évènements:
		
		\begin{description}
			\item[usager\_monte\_prioritaire] est l'évènement qui représente le fait qu'un passager prioritaire monte dans le bus.
			
			Cet évènement ne peut se produire que si le bus est arrêté, qu'au moins un usager prioritaire souhaite monter et que tous les passagers voulant descendre l'ont fait.
			
			L'effet de cet évènement est de déplacer l'usager prioritaire attendant à l'arrêt vers l'ensemble des passagers du bus.\\
			
			\item[usager\_monte\_non\_prioritaire] est l'évènement qui représente le fait qu'un passager non prioritaire monte dans le bus.
			
			Cet évènement ne peut se produire que si le bus est arrêté, qu'au moins un usager non prioritaire souhaite monter, que tous les passagers voulant descendre l'ont fait et que tous les passagers prioritaires sont montés.
			
			L'effet de cet évènement est de déplacer l'usager non prioritaire attendant à l'arrêt vers l'ensemble des passagers du bus.\\
		\end{description}
		
		Nous avons rajouté un dernier évènement permettant à notre simulation de ne pas se terminer. Celui-ci s'appelle \textbf{idle}. Il est appelé en boucle à partir du moment où la simulation se termine ou plus précisément, après l'évènement \textbf{bus\_repart}. Cela nous permet d'ignorer les deadlocks hors simulation. 
		
		

\subsection{DeadLock Freeness}
	Afin de prouver que notre premier raffinement n'a pas de situations	de blocages évidentes, nous avons vérifié le modèle à l'aide du plug-in "ProB". Vous pouvez voir nos résultats sur l'image ci-dessous. Afin de garantir ce résultat et d'éviter toutes les deadlocks, nous avons rajouté un théorème ou DeadLock Freeness. Nous avons réussi à le prouver, ce qui certifie que cette machine n'a aucun blocage. \\
	
	\includegraphics[scale=0.8]{checkMachineBus1.png}
	
\newpage
\section{MachineBus2}	
	La \textbf{MachineBus2} est le second raffinement de la machine de base \textbf{MachineBus0}. Elle intègre la notion de file sur les usagers non prioritaires.\\

	\subsection{Les évènements}
		On retrouve dans ce raffinement la plupart des évènements de la machine précédente.\\
		
		\begin{description}
			\item[usager\_monte\_non\_prioritaire] est l'évènement représentant un usager non prioritaire qui monte dans le bus.
			Cet évènement peut se produire lorsque le bus est à l'arrêt, qu'il n'y a aucun passager voulant descendre, que le bus n'est pas plein, qu'il y a des usagers non prioritaires voulant monter, qu'il n'y a pas d'usagers prioritaires voulant monter et que l'usager voulant monter est le plus ancien usager non prioritaire attendant à l'arrêt.
			L'effet de l'évènement est d'ajouter l'usager à la liste des passagers du bus, de l'enlever de la liste des usagers à l'arrêt et d'incrémenter l'indice de tête de file \textbf{oldest}.
			
			\item[usager\_arrive\_non\_prioritaire] est l'évènement représentant un usager non prioritaire qui arrive à l'arrêt de bus.
			Cet évènement peut se produire lorsque l'usager n'est ni est un passager et qu'il n'est pas dans la file des usagers non prioritaires attendant à l'arrêt.
			L'effet de l'évènement est d'ajouter l'usager à la liste des usagers non prioritaires attendant à l'arrêt ainsi que d'incrémenter l'indice de fin de file \textbf{newest}.

		\end{description}
		
	\subsection{Explication de la file de priorité}
		Nous avons décidé d'implémenter la liste en FIFO (First In First Out) présentée dans sur le site de M. Casteran. Nous avons compris son fonctionnement et nous l'avons adaptée à ce projet. Lorsqu'un usager non prioritaire arrive à l'arrêt, il est ajouté à la liste des personnes non prioritaires attendant à l'arrêt et l'indice de fin de file, \textbf{newest}, est incrémenté. Nous avons un second indice, \textbf{oldest}, qui nous permet de connaître le premier usager de la file. A chaque fois qu'un de ces usagers monte dans le bus, celui-ci est déplacé de l'ensemble des usagers non prioritaires attendant à l'arrêt vers l'ensemble des passagers du bus. Ensuite, l'indice \textbf{oldest} est incrémenté, afin qu'il pointe toujours sur le plus ancien usager non prioritaire attendant à l'arrêt.

\subsection{DeadLock Freeness}
	Afin de prouver que notre second raffinement n'a pas de situations de blocages évidentes, nous avons vérifié le modèle à l'aide du plug-in "ProB". Vous pouvez voir nos résultats sur l'image ci-dessous. Nous nous sommes contenté de ces résultats car il n'était pas demandé de réaliser la preuve pour ce raffinement. \\
	
	\includegraphics[scale=0.8]{checkMachineBus2.png}

\end{document}
