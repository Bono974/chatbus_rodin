\documentclass[a4paper,titlepage]{report}
\usepackage[T1]{fontenc}
\usepackage[utf8]{inputenc}
\usepackage[francais]{babel}
\usepackage{lmodern} % police véctorielle
\usepackage{pdfpages}
\usepackage{graphicx}
\usepackage{subfig}
\usepackage[hidelinks]{hyperref}
%\usepackage{url}
%\usepackage{array}
\usepackage{listings}
\usepackage{color}
% definition des marges
\usepackage[top=2.5cm,bottom=2.5cm,right=2.5cm,left=3.5cm]{geometry}
\usepackage{listingsutf8}
\usepackage{lscape}
\pagestyle{headings}

%retrait du numero des chapitre dans les sections
\makeatletter
\renewcommand{\thesection}{\@arabic\c@section}
\makeatother

\title{Rapport du projet de conception formelle :
	\begin{center}
		\textsc{Rodin 2014}
	\end{center}
}
\author{Jordan Aupetit, Salmane Bah, Timothée Sollaud \& Bruno Thiaolayel\\
	\\
	\\ \textsc{Université de Bordeaux}
}
\begin{document}
	\maketitle
	%\newpage
	\tableofcontents
	\newpage
	\thispagestyle{empty}	
	\newpage
	\pagenumbering{arabic} % Arabic page numbers from now on
% === Contenu =======================================================
\markboth{}{}
\section{Introduction}
	

\section{ContextUsagers}
	Nous avons défini un contexte usagers, afin de représenter l'ensemble fini des usagers et la capacité du bus.\\

\section{MachineBus0}
	La \textbf{MachineBus0} représente notre machine de base. Celle-ci assure la cohérence de l'enchaînement des événements et le respect de la règle de civisme proposée par le sujet. Ainsi, les usagers montent dans le bus qu'après que tous les passagers désirant descendre l'aient fait.\\
	
	\subsection{Les ensembles}
		\begin{description}
			\item[usagers\_a\_l\_arret] est un sous ensemble des usagers représentant les usagers attendant au prochain arrêt.\\
			\item[passagers] est un sous ensemble des usagers représentant les passagers du bus. Le cardinal de passager ne peut donc pas dépasser la capacité du bus.\\
			\item[passagers\_descendre] est un sous ensemble des passagers représentant les passagers souhaitant descendre du bus au prochain arrêt.\\
		\end{description}
		
	\subsection{Les événements}
		Pour définir les gardes de nos événements, nous avons utilisé le booléen \textbf{bus\_roule}, qui est à vrai si le bus est en mouvement et à faux si le bus est à un arrêt.\\
		
		\begin{description}
			\item[bus\_arrive] est l'événement qui représente le fait que le bus se stoppe à un arrêt.
			
			Cet événement peut se produire uniquement si le bus roule. De plus, afin de ne pas s'arrêter pour rien, on s'assure que des usagers souhaitent monter ou que des passagers veulent descendre et enfin, qu'il y a de la place dans le bus. Ainsi lorsque des usagers attendent le bus à un arrêt, si celui-ci est plein et que aucun passager ne souhaite descende, il ne s'arrêtera pas.
			
			L'effet de l'événement est simplement d'arrêter le bus.\\
		
			\item[bus\_repart] est l'événement qui représente le fait que le bus repart d'un arrêt.
			
			Cet événement peut se produire uniquement si le bus était arrêté. De plus, le bus ne doit repartir qu'une fois que tous les passagers souhaitant descendre l'ont fait et que les usagers attendant à l'arrêt soient tous montés ou que le bus soit plein.
			
			L'effet de l'événement est simplement de faire redémarrer le bus.\\
			
			\item[passager\_desc] est l'événement qui représente le fait qu'un passager descende du bus.
			
			Cet événement peut se produire uniquement si le bus est arrêté et qu'il y a au moins un passager qui souhaite descendre.
			
			L'effet de l'événement est de repasser le passager descendu en simple usager. \\
			
			\item[usager\_monte] est l'événement qui représente le fait qu'un passager monte dans le bus.
			
			Cet événement ne peut se produire que si le bus est arrêté, qu'au moins un usager souhaite monter et que tous les passagers voulant descendre l'ont fait.
			
			L'effet de cet événement est de convertir l'usager attendant à l'arrêt en passager du bus.\\
			
			\item[usager\_arrive] est l'événement représentant un usager arrivant à un arrêt, dans l'espoir de prendre le prochain bus.
			
			Cet événement peut se produire à tout moment, tant qu'il reste au moins un usager qui n'est ni à un arrêt, ni déjà passager du bus.
			
			L'effet de l'événement est simplement d'ajouter l'usager à la liste des usagers souhaitant monter.\\
			
			\item[passager\_veut\_desc] est l'événement qui représente le fait qu'un passager du bus veuille descendre au prochain arrêt.
			
			Cet événement peut se produire uniquement si le bus roule et qu'il reste des passagers n'ayant pas déjà décidé de descendre.
			
			L'effet de l'événement est d'ajouter le passager à l'ensemble des passagers souhaitant descendre.\\
		\end{description}
		
\section{MachineBus1}
	La \textbf{MachineBus1} est le premier raffinement de la machine de base \textbf{MachineBus0}. Elle intègre la gestion d'usagers prioritaires.\\
		
	\subsection{Les ensembles}
		Dans cette machine, les ensembles décrits précédemment n'ont pas été modifiés, excepté \textbf{usagers\_à\_l\_arrêt}, qui se décompose à présent en deux ensembles afin de gérer des usagers prioritaires :
		 	
		\begin{description}
			\item[usagers\_a\_l\_arret\_prioritaire], qui représente la partition des usagers prioritaires attendant au prochain arrêt.
			\item[usagers\_a\_l\_arret\_non\_prioritaire], qui représente la partition des usagers non prioritaires attendant au prochain arrêt.\\
		\end{description}
				
	\subsection{Les événements}
		On retrouve dans ce raffinement la plupart des événements de la machine d'origine, dans lesquels l'ensemble \textbf{usagers\_a\_l\_arret} a été remplacé par l'union des ensembles \textbf{usagers\_a\_l\_arret\_prioritaire} et \textbf{usagers\_a\_l\_arret\_non\_prioritaire}.\\
		
		L'événement \textbf{usager\_arrive} a été décomposé en deux événements, afin de gérer les priorités.
		\begin{description}
			\item[usagers\_arrive\_prioritaire] est l'événement représentant un usager prioritaire arrivant à un arrêt, dans l'espoir de prendre le prochain bus.
			
			Cet événement peut se produire à tout moment, tant qu'il reste au moins un usager qui n'est ni en attente à un arrêt (prioritaire ou non), ni déjà passager du bus.
			
			L'effet de l'événement est simplement d'ajouter l'usager à la liste des usagers prioritaires souhaitant monter.\\
			
			\item[usagers\_monte\_non\_prioritaire] est l'événement représentant un usager non prioritaire arrivant à un arrêt, dans l'espoir de prendre le prochain bus.
			
			Cet événement peut se produire à tout moment, tant qu'il reste au moins un usager qui n'est ni en attente à un arrêt (prioritaire ou non), ni déjà passager du bus.
			
			L'effet de l'événement est simplement d'ajouter l'usager à la liste des usagers non prioritaires souhaitant monter.\\
		\end{description}
		
		De même, afin de gérer les priorités, l'événement \textbf{usager\_monte} a été décomposé en deux événements.
		\begin{description}
			\item[usagers\_monte\_prioritaire] est l'événement qui représente le fait qu'un passager prioritaire monte dans le bus.
			
			Cet événement ne peut se produire que si le bus est arrêté, qu'au moins un usager prioritaire souhaite monter et que tous les passagers voulant descendre l'ont fait.
			
			L'effet de cet événement est de convertir l'usager prioritaire attendant à l'arrêt en passager du bus.\\
			
			\item[usagers\_monte\_non\_prioritaire] est l'événement qui représente le fait qu'un passager non prioritaire monte dans le bus.
			
			Cet événement ne peut se produire que si le bus est arrêté, qu'au moins un usager non prioritaire souhaite monter, que tous les passagers voulant descendre l'ont fait et que tous les passagers prioritaires sont montés.
			
			L'effet de cet événement est de convertir l'usager non prioritaire attendant à l'arrêt en passager du bus.\\
		\end{description}
			
\end{document}
